% Created 2020-02-17 Mon 12:43
% Intended LaTeX compiler: pdflatex
\documentclass[11pt]{article}
\usepackage[utf8]{inputenc}
\usepackage[T1]{fontenc}
\usepackage{graphicx}
\usepackage{grffile}
\usepackage{longtable}
\usepackage{wrapfig}
\usepackage{rotating}
\usepackage[normalem]{ulem}
\usepackage{amsmath}
\usepackage{textcomp}
\usepackage{amssymb}
\usepackage{capt-of}
\usepackage{hyperref}
\author{dustin}
\date{\today}
\title{}
\hypersetup{
 pdfauthor={dustin},
 pdftitle={},
 pdfkeywords={},
 pdfsubject={},
 pdfcreator={Emacs 28.0.50 (Org mode 9.3.2)}, 
 pdflang={English}}
\begin{document}

\tableofcontents

\section{SANDER-NEXMD Installation}
\label{sec:org5b20f8a}
Installation of the SANDER-NEXMD installation will require the following as well
as their dependencies:

\begin{itemize}
\item NEXMD (Git Access)
\item Amber (Git Access)
\end{itemize}

We also strongly encourage the use of the intel/2017 compiler w/ mkl. GNU is
supported but users will have better performance with the intel compiler.

First install NEXMD. Access to the NEXMD git is restricted. Please contact
either Adrian Roitberg (roitberg@ufl.edu) or Sergei Tretiak (serg@lanl.gov) for
permission.

CD to your application director, then pull the NEXMD git using. 

\begin{verbatim}
git clone https://github.com/roitberg/nexmd.git
\end{verbatim}


And check-in to the amber-nexmd-na branch.

\begin{verbatim}
cd nexmd 
git checkout amber-nexmd-na 
\end{verbatim}

Try to install the standalone NEXMD program. Activate your chosen compiler, then use either: 

\begin{verbatim}
make ic_mkl # For intel w/ mkl 
make gnu_mkl # For gnu w/ mkl 
make gnu # For gnu wo/ mkl (not recommended) 
\end{verbatim}

If this fails, refer to the documentation included in the NEXMD git. Please contact either Dustin Tracy (dtracy@ufl.edu) or Benjamin Nebgen (bnebgen@lanl.gov) for further questions regarding compilation of NEXMD. Once we’ve successfully installed NEXMD, we need to flush the standalone temp files so AMBER can build the Library files. 

\begin{verbatim}
make clean 
\end{verbatim}

Now we need to set a variable so AMBER knows where to find NEXMD. 

\begin{verbatim}
export NAESMDHOME=`pwd` 
\end{verbatim}

Note that currently NAESMDHOME is used here, not NEXMDHOME. The new name will be
updated soon. We are now done with prepping NEXMD.

We now need to install AMBERTOOLS. We are currently using a restricted git for
development. Please contact Dustin Tracy (dtracy@ufl.edu) for access to a tar
file.

CD into your application directory and extract the tar file. 

\begin{verbatim}
tar -xvf amber_na.tar.gz 
\end{verbatim}

Try to build the standalone AMBER program.  

\begin{verbatim}
cd amber 
export AMBERHOME=`pwd` 

./configure –noX11 --skip-python -norism  intel mkl # for intel  w/ mkl 

./configure –noX11 --skip-python -norism  gnu mkl # for gnu  w/ mkl 

./configure –noX11 --skip-python -norism  gnu # for gnu 
\end{verbatim}

Use the same configuration you used for your NEXMD build. 

Then install 

\begin{verbatim}
source ./amber.sh 

make install 
\end{verbatim}

If this fails please refer to the amber manual (\url{https://ambermd.org/Manuals.php})
for troubleshooting and contact information. Once successful installation and
testing of the standalone AmberTools packages is completed cd back into
AMBERHOME and configure AMBER to link to the NEXMD package.

\begin{verbatim}
cd $AMBERHOME 

./configure –noX11 --skip-python -norism -naesmd intel mkl # for intel  w/ mkl 

./configure –noX11 --skip-python -norism -naesmd gnu mkl # for gnu  w/ mkl 

./configure –noX11 --skip-python -norism -naesmd gnu # for gnu 
\end{verbatim}

We don’t need to rebuild all AMBER now, only SANDER, so we CD into SANDER and build from there. 

\begin{verbatim}
cd AmberTools/src/sander 

make install 
\end{verbatim}

To test successful installation (python3 required) 

\begin{verbatim}
cd $NAESMDHOME/testscripts/amber_nexmd_testscripts 

python run_tests.py 
\end{verbatim}

\section{SANDER-NEXMD Single Trajectory Run}
\label{sec:orgb1fd73f}

Though our implementation of the simulation of non-adiabatic dynamics requires a
large number of independent trajectories, the SANDER-NEXMD interface can only
manage a single trajectory. Another script (to be introduced later) controls the
characteristics of the swarm. Each single SANDER-NEXMD trajectory will require a
number of inputs.

\begin{itemize}
\item Amber Input (mdin)

\item NEXMD Input (input.ceon)

\item Amber Coordinate File (mdcrd)

\item Amber Parameter File (prmtop)
\end{itemize}

Full examples can be found in \$NAESMDHOME/tests/amber\textsubscript{nexmd}/. Refer to the the
included README in that directory to determine the type of trajectory for each
test.

Amber Input 

\begin{verbatim}
300K constant temp QMMMMD 
 &cntrl 
  ** Normal Amber Input (Check manual) ** 
  ifqnt=1 
 / 

 &qmmm 
  verbosity=5, 
  qmmask=':1', 
  nae=1 
 / 
\end{verbatim}

No special behavior is needed from cntrl besides the activation of qm/mm (setting ifqnt=1). Most QM/MM behavior is controlled by the input.ceon file for the exception of the verbosity, the atoms the include in the qm calculations (qmmask) and the referral to nexmd (nae). Please refer the AMBER manual for information regarding the parameters found in \&cntrl. 

NEXMD Input 

\begin{verbatim}
&qmmm 
  maxcyc=0, ! Optimization must be turned off
 ** Normal NEXMD Input **  
 ** Include all of AMBER’s qm/mm flags besides verbosity, qmmask ** 
&endqmmm 

&moldyn 
  ** Normal NEXMD Input ** 
&endmoldyn 
&coord 
  ** Block must be included but is ignored ** 
&endcoord 
&veloc 
  ** Block must be included but is ignored ** 
&endveloc 
&coeff 
  ** Normal NEXMD Input ** 
&endcoeff 
\end{verbatim}

The NEXMD-SANDER interface is designed to be able to read an unmodified nexmd
file. For most users, little to no modification will be required beyond setting
the qmewald parameters. Note that the initial coordinates and velocities are
read from the amber intput files and any values included in the input.ceon file
will be ignored.

\subsection{AMBER Coordinate File}
\label{sec:orgaaf5152}

Amber coordinate files are needed to run the SANDER-NEXMD interface. These can
be created using tleap, or through a converter built into PyNASQM. To use the
pynasqm converter

\begin{verbatim}
amber-nexmd-converter.py input.ceon mdcrd
\end{verbatim}

Further instruction for prepping a job can be found in the Creating a System
section.

\subsection{AMBER prmtop}
\label{sec:org91f68d5}

AMBER prmtop files can be generated using tleap. Refer to the AMBER manual for
this procedure or look at the Create a QM/MM System for SANDER-NEXMD.

\section{Creating a QM/MM System for SANDER-NEXMD\hfill{}\textsc{METHOD}}
\label{sec:orge750ba6}
\begin{enumerate}
\item Build your solute using either Avogadro or Gaussview
\item Create pepi files for each using antechamber that came with AMBER for each of
the following commands \texttt{\$molecule=molecule name} 
\begin{verbatim}
antechamber -fi pdb -fo prepi -i $molecule.pdb -o o2.prepi
\end{verbatim}
\item create frcmod files for each
\begin{verbatim}
parmchk2 -f prepi -i $molecule.prepi -o o2.frcmod
\end{verbatim}
\item create mol2 file with
\begin{verbatim}
antechamber -fi pdb -fo mol2 -i $molecule.pdb -o o2.mol2 -rn o2 -c bcc -pf y
\end{verbatim}
\item run tleap with
\begin{verbatim}
cat << EOF > leap.in
source leaprc.gaff
source leaprc.water.tip3p
loadamberparams $solute.frcmod
loadamberparams $solution.frcmod
$solute=loadmol2 $solute.mol2
$solution=loadmol2 $solution.mol2
solvatebox $solute $solution 30
saveamberparm $solute $solute.prmtop $solute.inpcrd
quit
EOF
tleap -f leap.in
\end{verbatim}
Note that the line \texttt{solvatebox \$solute \$solution 30} is the size of the
 box, you can change this to anything but the system will crash if the box
 isn't twice the length of the QM box.
\item You should now have the following 2 files. A parameter files \texttt{\$solute.prmtop}
and \texttt{\$solute.inpcrd}. We now want to equilibrate this system. Create the
following files
\begin{verbatim}
m1_min1.in
\end{verbatim}

\begin{verbatim}
initial minimization solvent + ions
&cntrl
  imin   = 1,
  maxcyc = 1000,
  ncyc   = 500,
  ntb    = 1,
  ntr    = 1,
  cut    = 10.0
/
Hold Solute fixed
500.0
ATM 1 <number of solute atoms>
END
END
\end{verbatim}
Note : The line \texttt{ATM 1 <number of solute atoms>} should be from 1 to Number of atoms in solute.

\begin{verbatim}
m1_min2.in
\end{verbatim}

\begin{verbatim}
initial minimization solvent + ions
&cntrl
  imin   = 1,
  maxcyc = 2500,
  ncyc   = 1000,
  ntb    = 1,
  ntr    = 0,
  cut    = 10.0
/
\end{verbatim}

\begin{verbatim}
m1_md1.in
\end{verbatim}

\begin{verbatim}
MD Equilibration STEP
&cntrl
  imin   = 0,
  irest  = 0,
  ntx    = 1,
  ig     =-1,
  ntb    = 1,
  cut    = 10.0,
  ntr    = 1,
  ntc    = 2,
  ntf    = 2,
  tempi  = 0.0,
  temp0  = 300.0,
  ntt    = 3,
  gamma_ln = 2.0,
  nstlim = 5000, 
  dt = 0.002,
  ntpr = 100,
  ntwx = 100,
  ntwr = 1000
/
Keep fixed with weak restraints
10.0
ATM 1 <number of solute atoms>
END
END
\end{verbatim}
Note : The line \texttt{ATM 1 <number of solute atoms>} should be from 1 to Number of atoms in solute.

\begin{verbatim}
m_md2.in
\end{verbatim}

\begin{verbatim}
Constant Pressure Relaxation
&cntrl
  imin = 0, 
  irest = 1,
  ntx = 5,
  ntb = 2,
  pres0 = 1.0,
  ntp = 1,
  ig = -1,
  taup = 2.0,
  cut = 10.0,
  ntr = 0,
  ntc = 2,
  ntf = 2,
  tempi = 300.0,
  temp0 = 300.0,
  ntt = 3,
  gamma_ln = 2.0,
  nstlim = 100000,
  dt = 0.002,
  ntpr = 100,
  ntwx = 100,
  ntwv = -1,
  ntwr = 1000
/
\end{verbatim}
\item Now we want to create our equilibrated system create a file
\begin{verbatim}
box_eq.sh
\end{verbatim}

\begin{verbatim}
echo 'm1_min'
sander -O -i m1_min.in -o m1_min.out -r m1_min.rst -c m1.inpcrd -p m1.prmtop -ref m1.inpcrd
echo 'm1_min1'
sander -O -i m1_min2.in -o m1_min2.out -r m1_min2.rst -c m1_min.rst -p m1.prmtop
echo 'm1_md1'
sander -O -i m1_md1.in -o m1_md1.out -r m1_md1.rst -c m1_min2.rst -p m1.prmtop -ref m1_min2.rst
echo 'm1_md2'
sander -O -i m1_md2.in -o m1_md2.out -r m1_md2.rst -c m1_md1.rst -p m1.prmtop
echo 'finished'
\end{verbatim}
\item Run this with
\begin{verbatim}
./box_eq.sh
\end{verbatim}
This will leave you with an equilibrated geometry file \texttt{m1\_md2.rst}.
\item We now only need the NEXMD input and SANDER input file to begin. 
\begin{verbatim}
md_qmmm_amb.in
\end{verbatim}

\begin{verbatim}
300K constant temp QMMMMD
&cntrl
  imin=0,
  iwrap=1,
  irest=0,
  ntx=5,
  ntb=1,
  ntp=0,
  ig=-1,
  taup=2.0,
  cut=16.0,
  ntr=0,
  tempi=300.0,
  temp0=300.0,
  ntt=3, ! Use Langevin
  gamma_ln=2.0, ! Lavenvin constant
  nstlim=20000, ! Number of Step
  dt=0.0005,
  ntpr=10, ! print every 10 steps
  ntwx=10, ! print coords every 10 steps
  ntwv=-1, ! save velocities every time coords are saved
  ifqnt=1 ! Do QM calculations
/
&qmmm
  verbosity=1,
  qmmask=':1', ! Only use QM on the solute
  nae=1 ! Activate NEXMD Looks for input.ceon file
/
\end{verbatim}
\begin{verbatim}
input.ceon
\end{verbatim}

\begin{verbatim}
&qmmm
  qm_theory='AM1',
  diag_routine=1,
  qmcharge=0,
  qmshake=0,
  qm_ewald=0,
  qm_pme=0,
  scfconv=1.0000E-10,
  printcharges=1,
  printdipole=0,
  printbondorders=0,
  density_predict=0,
  itrmax=300,
  diag_routine=1,
  exst_method=2,
  dav_guess=0,
  ftol0=1.0000E-05, ! Acceptance Tolerance for Davidson (emin-eold)
  ftol1=1.0000E-06, ! Acceptance Tolerance for Davidson (residual)
  dav_maxcyc=200,
  calcxdens=.false.,
  maxcyc=0,
  ntpr=1,
  grms_tol=1.0000E-02,
  solvent_model=0,
  potential_type=1,
  ceps=10,
  linmixparam=1,
  cosmo_scf_ftol=1.0000E-05,
  EF=0,
  Ex=0.0000E+00,
  Ez=0.0000E+00, 
  Ey=0.0000E+00,  !1.000E-02
  onsager_radius=2,
&endqmmm
&moldyn
  !***** General parameters
  rnd_seed=1, ! seed for the random number generator
  bo_dynamics_flag=1, ! 0-non-BO, 1-BO [1]
  exc_state_init=0, ! initial excited state (0 - ground state) [0]
  n_exc_states_propagate=0, ! number of excited states [0]

  !***** Dynamics parameters
  time_init=0.d0, ! initial time, fs [0.0]
  time_step=0.5, !time step, fs [0.1]
  n_class_steps=0, !number of classical steps [1]
  n_quant_steps=0, ! number of quantum steps for each classical step [4]
  moldyn_deriv_flag=1, ! 0-none, 1-analyt, 2-numeric [1]
  num_deriv_step=1.d-5, ! displacement for numerical derivatives, A [1.d-3]
  rk_tolerance=1.d-7, ! tolerance for the Runge-Kutta propagator [1.d-7]

  !***** Non-adiabatic parameters
  quant_step_reduction_factor=2.5d-2, ! quantum step reduction factor [0.1]
  decoher_type=2, ! type of decoherence: Persico/Granucci (0), Truhlar(1) [0]
  decoher_e0=0.d0, ! decoherence parameter E0 [0.1]
  decoher_c=0.d0, ! decoherence parameter C [0.1]
  dotrivial=1

  !***** Thermostat parameters
  therm_type=1, ! Thermostat type (0-no thermostat,1-Langevin,2-Berendsen) [0]
  therm_temperature=300.d0, ! Thermostate temperature, K [300.0]
  therm_friction=2.d0, ! thermostate friction coefficient, 1/ps [2.0]
  berendsen_relax_const=0.4d0, ! bath relaxation constant, only for Berendsen [0.4]
  heating=0, ! heating (1) or equilibrated(0) [0]
  heating_steps_per_degree=100, ! number of steps per degree during heating [100]

  !***** Output & Log parameters
  verbosity=3, ! output verbosity (0-minimal, 3-highest) [2]
  out_data_steps=100, ! number of steps to write data [1]
  out_coords_steps=100, ! number of steps to write the restart file [10]
  out_data_cube=0, ! write(1) or not(0) view files to generate cubes [0]
  out_count_init=0, ! the initial count for output files [0]
&endmoldyn

&coord
&endcoord

&veloc
&endveloc

&coeff
      0.0000000000000000       0.0000000000000000
      0.0000000000000000       0.0000000000000000
&endcoeff

\end{verbatim}

Note that the coord and veloc must be there, but there values will be
overridden by amber during dynamics.
\item Run the md with the command
\begin{verbatim}
sander -O -i md_qmmm_amb.in -o mdout -p m1.prmtop -c m1_md2.rst -x traj_file.nc
\end{verbatim}
\end{enumerate}
\section{Using Pynasqm for Multiple Trajectories}
\label{sec:orge60bd1c}
\subsection{Installation}
\label{sec:org705388a}
Clone this repository to the directory of your choosing.
\begin{verbatim}
git clone https://github.com/PotentialParadox/pynasqm.git
cd pynasqm
python setup install
\end{verbatim}
\subsection{Usage}
\label{sec:orgb5143c1}
Pynasqm is currently capable of automating the following actions
\begin{itemize}
\item Single MM-Ground State Trajectory
\item QM-Ground State Trajectories originating from MM-Ground-State
\item Absorption calculations using snapshots from the QM-Ground State Trajectories
\item PulsePump SinglePoint Calculations
\item QM-Excited State Trajectory
\begin{itemize}
\item Franck-Condon Excitations
\item Pule Pump Simulations
\item Capable of performing Non-Adiabatic using the isTully option
\end{itemize}
\end{itemize}

Follow instructions for Creating a QM/MM System for SANDER-NEXMD to create an
AMBER parameter/topology (prmtop) file and an AMBER coordinate (inpcrd) file.
Create a new directory with any name you'd like.
Copy your prmtop and inpcrd file to that directory, renaming the files to 
\begin{itemize}
\item m1.prmtop (prmtop file)
\item m1\textsubscript{md2.rst} (restart file)
\end{itemize}
You can initialize the other input files with
\begin{verbatim}
nasqm.py --init
\end{verbatim}
which will generate
\begin{itemize}
\item pynasqm.in (pynasqm input file)
\item input.ceon (The NEXMD input file)
\item md\textsubscript{qmmm}\textsubscript{amb.in} (The AMBER input file)
\end{itemize}

For the documentation for NEXMD, please refer to \href{https://github.com/roitberg/nexmd}{NEXMD-Git} (Contact either
Adrian Roitberg (roitberg@ufl.edu) or Sergei Tretiak (serg@lanl.gov) for
permission). For the AMBER input file, refer to \href{https://ambermd.org/doc12/Amber19.pdf}{AMBER-Manual}. 

The pynasqm input file is written as a commented J-SON file. On lines other than the ones
commented out, the file is white space sensitive. Any edits should be performed
within the second quoted value on the line. Refer to the comments within the
file for guidance on the inputs. Any further question can be referred to Dustin
Tracy (dtracy@ufl.edu).
systems employing the slurm array feature.
\end{document}